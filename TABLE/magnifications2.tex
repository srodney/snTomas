\begin{deluxetable*}{p{0.62in}rrrrp{1.4in}p{2.5in}}
\tablecolumns{7}
\tablecaption{Tested lens models for Abell 2744.\label{tab:LensModels}}
\tablehead{ \colhead{Model} & \colhead{N$_{\rm sys}$\tablenotemark{a}} & \colhead{N$_{\rm im}$\tablenotemark{b}} & 
            \colhead{N$_{\rm spec}$\tablenotemark{c}} & \colhead{N$_{\rm phot}$\tablenotemark{d}} & 
            \colhead{References} & \colhead{Description}}
\startdata
Bradac(v1)     & 16 &  56 & 2 & 14 &  HFF\tablenotemark{e}; \citealt{Bradac:2009} & {\tt SWUnited}\tablenotemark{f} : Free-form, strong+weak-lensing based model. Errors from bootstrap resampling only weak-lensing constraints.\\
CATS(v1)       & 17 &  60 & 2 & 14 &  HFF; \citealt{Richard:2014} &  CATS\tablenotemark{g} team implementation of LENSTOOL\tablenotemark{h} parametric strong-lensing based model.\\
CATS(v1sw)     & 17 &  60 & 2 & 14 &  \citealt{Richard:2014} &  CATS team LENSTOOL parametric model with both strong and weak lensing constraints.\\
Merten         & 16 &  56 & 2 & 14 &  HFF; \citealt{Merten:2011} &  {\tt SaWLENS},\tablenotemark{i} Grid-based free-form strong+weak lensing based model using adaptive mesh refinement.\\
Sharon(v1)     & 17 &  60 & 2 & 14 &  HFF;  & LENSTOOL parametric, strong-lensing based model\\
Sharon(v2)     & 15 &  47 & 3 & 11 &  \citealt{Johnson:2014} & LENSTOOL parametric, strong-lensing based model. Includes cosmological parameter variations in uncertainty estimates.\\
Zitrin-LTM     & 10 &  44 & 2 & 0  &  HFF; \citealt{Zitrin:2009a} & Parametric strong-lensing model, adopts the Light-Traces-Mass assumption for both the luminous and dark matter.\\
Zitrin-NFW     & 10 &  44 & 2 & 0  &  HFF; \citealt{Zitrin:2013a} &  Parametric strong-lensing model using PIEMD profiles for galaxies and NFW profiles for dark matter halos.\\
Williams       & 10 &  40 & 2 & 8  &  HFF & {\tt GRALE}\tablenotemark{j} : Free-form strong-lensing model using a genetic algorithm.  \\
\cutinhead{Post-HFF models : include data from the HFF program}\\
CATS(v2)       & 50 & 151 & 4 & 1  &  \citealt{Jauzac:2014c} & Updated version of the CATSv1 model, adds 33 new multiply-imaged galaxies, for a total of 159 individual lensed images.\\
Diego\tablenotemark{k} & 15 &  48 & 4 & 11  & \citealt{Diego:2014b} & {\tt WSLAP+}\tablenotemark{l} : Free-form strong-lensing model using a grid-based method, supplemented by deflections fixed to cluster member galaxies.\\
GLAFIC         & 24 &  67 & 3 & 12 &  \citealt{Ishigaki:2015} & Parametric strong-lensing model using v1.0 of the {\tt GLAFIC} code.\tablenotemark{m} \\
Lam(v1)        & 21 &  65 & 4 & 17 &  \citealt{Lam:2014} & Alternative implementation of the {\tt WSLAP+} model, using a different set of strong-lensing constraints and redshifts.\\
\cutinhead{Unblind models : generated after the SN magnification was known}\\
Bradac(v2)     & 25 &  72 & 7 & 18 &  \citealt{Wang:2015} & Updated version of the {\tt SWUnited} model with new strong-lensing constraints from HFF imaging and GLASS spectra. Errors from bootstrap resampling only strong-lensing constraints.\\
Lam(v2)        & 10 &  32 & 5 & 5  &  \citealt{Lam:2014} & Updated version of the {\tt WSLAP+} model, using more selective strong-lensing constraints and GALFIT\tablenotemark{n} models for galaxy mass.\\
CATS(v2.1)     & 55 & 154 & 8 & 1  &  \citealt{Jauzac:2014c} & Updated version of the CATSv2 model, adopting the spec-z constraints used for the Bradac(v2) model.\\
CATS(v2.2)     & 25 &  72 & 8 & 1  &  \citealt{Jauzac:2014c} & Updated version of the CATSv2 model, adopting the spec-z constraints and multiple-image definitions used for the Bradac(v2) model.
\enddata
\tablenotetext{a}{Number of multiply imaged systems used as strong-lensing constraints.}
\tablenotetext{b}{Total number of multiple images used.}
\tablenotetext{c}{Number of multiply imaged systems with spectroscopic redshifts.}
\tablenotetext{d}{Number of multiply imaged systems with photometric redshifts.}
\tablenotetext{e}{Lens models with the reference code ``HFF'' were produced as part of the 
Hubble Frontier Fields lens modeling
program, using arcs identified in HST archival imaging from
\citealt{Merten:2011}, spectroscopic redshifts from
\citealt{Richard:2014}, and ground-based imaging from
\citealt{Cypriano:2004,Okabe:2008,Okabe:2010a,Okabe:2010b}. Details on
the model construction and an interactive model magnification web interface 
are available at
\url{http://archive.stsci.edu/prepds/frontier/lensmodels/} 
}
\tablenotetext{f}{{\tt SWunited} : Strong and Weak lensing United; \citealt{Bradac:2005}}
\tablenotetext{g}{{\tt CATS} : Clusters As TelescopeS lens modeling team. PI's: J.-P. Kneib \&\ P. Natarajan}
\tablenotetext{h}{{\tt LENSTOOL} : \citealt{Jullo:2007}; \url{http://projects.lam.fr/repos/lenstool/wiki}}
\tablenotetext{i}{{\tt SaWLENS} : \citealt{Merten:2009}; Strong and Weak LENSing analysis code. \url{http://www.julianmerten.net/codes.html}}
\tablenotetext{j}{{\tt GRALE} : GRAvitational LEnsing; \citealt{Liesenborgs:2006,Liesenborgs:2007,Mohammed:2014}}
\tablenotetext{k}{Abell 2744 model available at \url{http://www.ifca.unican.es/users/jdiego/LensExplorer}. 
No uncertainty estimates were available for the Diego implementation of the {\tt WSLAP+} model, so we adopt the uncertainties from the closely related Lam model.
}
\tablenotetext{l}{{\tt WSLAP+} : \citealt{Sendra:2014}; Weak and Strong Lensing Analysis Package plus member galaxies (Note: no weak-lensing constraints used for Abell 2744)}
\tablenotetext{m}{{\tt GLAFIC} : \citealt{Oguri:2010}; \url{http://www.slac.stanford.edu/~oguri/glafic/}}
\tablenotetext{n}{{\tt GALFIT} : Two-dimensional galaxy fitting algorithm \citep{Peng:2002}}
\end{deluxetable*}




\begin{deluxetable}{p{0.62in}ccc}
\tablecolumns{4}
\tablecaption{Lens model predictions for SN \tomas\ magnification\label{tab:PredictedMagnifications}}
\tablehead{ \colhead{Model\tablenotemark{a}} & 
            \colhead{Best\tablenotemark{b}} & \colhead{Median\tablenotemark{c}} & 
            \colhead{68\% Conf. Range\tablenotemark{d}}}
\startdata
Bradac(v1)      & 3.19     & 2.48  &   2.31$-$2.66 \\
CATS(v1)        & 2.28     & 2.29  &   2.25$-$2.34 \\
CATS(v1sw)      & \nodata  & 2.62  &   2.44$-$2.80 \\
Merten          & 2.33     & 2.24  &   2.04$-$2.92 \\
Sharon(v1)      & 2.56     & 2.60  &   2.44$-$2.78 \\
Sharon(v2)      & 2.74     & 2.59  &   2.42$-$2.85 \\
Zitrin-LTM      & 2.67     & 2.99  &   2.61$-$3.77 \\
Zitrin-NFW      & 2.09     & 2.29  &   2.07$-$2.52 \\
Williams        & 2.70     & 2.81  &   1.65$-$5.54 \\[0.1em]
\tableline\\[-0.5em]
CATS(v2)        & \nodata  & 3.42  &   3.27$-$3.58 \\
Diego           &  \nodata  & 1.80  &   1.44$-$2.16 \\
GLAFIC          & 2.34     & 2.29  &   2.19$-$2.37 \\
Lam(v1)         & \nodata  & 2.79  &   2.42$-$3.16 \\[0.5em]
Bradac(v2)$^{\rm *}$ & 2.23     & 2.26  &   2.30$-$2.23 \\
Lam(v2)$^{\rm *}$ & 1.86     & 1.91  &   1.54$-$2.28\\
CATS(v2.1)$^{\rm *}$ & \nodata & 3.06  &   2.92$-$3.19 \\
CATS(v2.2)$^{\rm *}$ & \nodata & 3.07  &   2.94$-$3.20 \\
\enddata
\tablenotetext{a}{Models above the line are from the pre-HFF set, and those below incorporate HFF data. The final two, marked by an asterisk, were not part of the blind test as they include modifications made after the measured magnification of the SN was known.}  
\tablenotetext{b}{The magnification returned for the optimal version of each model, as independently defined by each lens modeling team.}  
\tablenotetext{c}{Median magnification from 100-600 Monte Carlo realizations of the model.}  
\tablenotetext{d}{Confidence ranges about the median, enclosing 68\%\ of the realized values.}
\end{deluxetable}
