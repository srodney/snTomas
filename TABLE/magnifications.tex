\begin{deluxetable*}{p{0.7in}cccp{1.4in}p{2.5in}}
\tablecolumns{6}
\tablecaption{Predicted magnifications for SN \tomas\ from lens models.\label{tab:PredictedMagnifications}}
\tablehead{ \colhead{Model} & \colhead{Best\tablenotemark{a}} & \colhead{Median\tablenotemark{b}} & \colhead{68\% Conf. Range\tablenotemark{c}} & \colhead{References} & \colhead{Description}}
\startdata
Sharon(v1)   & 2.56     & 2.60  &   2.44$-$2.78 &   HFF\tablenotemark{d}  & LENSTOOL\tablenotemark{e} parametric, strong-lensing based model\\
Sharon(v2)   & 2.74     & 2.59  &   2.42$-$2.85 &   HFF\tablenotemark{d}; \citealt{Johnson:2014} & LENSTOOL parametric, strong-lensing based model.\\
CATS(v1)     & 2.28     & 2.29  &   2.25$-$2.34 &   HFF\tablenotemark{d}; \citealt{Richard:2014} &  CATS\tablenotemark{f} team implementation of LENSTOOL parametric strong-lensing based model.\\
CATS(v1sw)   & \nodata  & 2.62  &   2.44$-$2.80 &   HFF\tablenotemark{d}; \citealt{Richard:2014} &  CATS\tablenotemark{f} team LENSTOOL parametric model with both strong and weak lensing constraints.\\
CATS(v2)     & \nodata  & 3.42  &   3.27$-$3.58 &   HFF\tablenotemark{d}; \citealt{Jauzac:2014} & Updated version of the CATSv1 model, adds 33 new multiply-imaged galaxies.\\
GLAFIC       & 2.34     & 2.29  &   2.19$-$2.37 &   HFF\tablenotemark{d}; \citealt{Ishigaki:2015} & Parametric strong-lensing model using v1.0 of the {\tt GLAFIC} code.\tablenotemark{g} \\
Zitrin-NFW   & 2.09     & 2.29  &   2.07$-$2.52 &   HFF\tablenotemark{d}; \citealt{Zitrin:2013a} &  Parametric strong-lensing model using PIEMD\tablenotemark{h} profiles for galaxies and NFW\tablenotemark{i} profiles for dark matter halos.\\
Zitrin-LTM   & 2.67     & 2.99  &   2.61$-$3.77 &   HFF\tablenotemark{d}; \citealt{Zitrin:2009a} & Parametric strong-lensing model, adopts the Light-Traces-Mass assumption for both the luminous and dark matter.\\
Bradac(v1)   & 3.19     & 2.48  &   2.31$-$2.66 &   HFF\tablenotemark{d}; \citealt{Bradac:2009} & {\tt SWUnited}\tablenotemark{j} : Free-form, strong+weak-lensing based model. Errors from bootstrap resampling only weak-lensing constraints.\\
Bradac(v2)   & 2.23     & 2.26  &   2.30$-$2.23 &   \citealt{Wang:2015} & Updated version of the {\tt SWUnited}\tablenotemark{j} model with new strong-lensing constraints from HFF imaging. Errors from bootstrap resampling only strong-lensing constraints.\\
Williams     & 2.70     & 2.81  &   1.65$-$5.54 &   HFF\tablenotemark{d} & {\tt GRALE}\tablenotemark{k} : Free-form strong-lensing model using a genetic algorithm.  \\
Merten       & 2.33     & 2.24  &   2.04$-$2.92 &   HFF\tablenotemark{d}; \citealt{Merten:2011} &  {\tt SaWLENS},\tablenotemark{l} Grid-based free-form strong+weak lensing based model using adaptive mesh refinement.\\
Lam          & \nodata  & 2.79  &   2.42$-$3.16 &   \citealt{Lam:2014} & {\tt WSLAP+}\tablenotemark{m} : Free-form model strong-lensing using a grid-based method, supplemented by deflections fixed to cluster member galaxies.\\
Diego\tablenotemark{n}  & \nodata  & 1.80  &   1.44$-$2.16 & \citealt{Diego:2014b} & Alternative implementation of the {\tt WSLAP+}\tablenotemark{m} model, using a different set of strong-lensing constraints and redshifts.
\enddata
\tablenotetext{a}{The magnification returned for the optimal version of each model, as independently defined by each lens modeling team.}  
\tablenotetext{b}{Median magnification from 100-600 Monte Carlo realizations of the model.}  
\tablenotetext{c}{Confidence ranges about the median, enclosing 68\%\ of the realized values.}
\tablenotetext{d}{Lens models with the reference code ``HFF'' were produced as part of the 
Hubble Frontier Fields lens modeling
program, using arcs identified in HST archival imaging from
\citealt{Merten:2011}, spectroscopic redshifts from
\citealt{Richard:2014}, and ground-based imaging from
\citealt{Cypriano:2004,Okabe:2008,Okabe:2010a,Okabe:2010b}. Details on
the model construction and an interactive model magnification web interface 
are available at
\url{http://archive.stsci.edu/prepds/frontier/lensmodels/} 
}
\tablenotetext{e}{{\tt LENSTOOL} : \citealt{Jullo:2007}; \url{http://projects.lam.fr/repos/lenstool/wiki}}
\tablenotetext{f}{{\tt CATS} : Clusters As TelescopeS lens modeling team. PI's: J.-P. Kneib \&\ P. Natarajan}
\tablenotetext{g}{{\tt GLAFIC} : \citealt{Oguri:2010}; \url{http://www.slac.stanford.edu/~oguri/glafic/}}
\tablenotetext{h}{PIEMD: Pseudo Isothermal Elliptical Mass Distrubition}
\tablenotetext{i}{NFW : Navarro-Frenk-White mass density profile \citep{Navarro:1997}.}
\tablenotetext{j}{{\tt SWunited} : Strong and Weak lensing United; \citealt{Bradac:2005}}
\tablenotetext{k}{{\tt GRALE} : GRAvitational LEnsing; \citealt{Liesenborgs:2006,Liesenborgs:2007,Mohammed:2014}}
\tablenotetext{l}{{\tt SaWLENS} : \citealt{Merten:2009}; Strong and Weak LENSing analysis code. \url{http://www.julianmerten.net/codes.html}}
\tablenotetext{m}{{\tt WSLAP+} : \citealt{Sendra:2014}; Weak and Strong Lensing Analysis Package plus member galaxies (Note: no weak-lensing constraints used for Abell 2744)}
\tablenotetext{n}{Abell 2744 model available at \url{http://www.ifca.unican.es/users/jdiego/LensExplorer}. 
No uncertainty estimates were available for the Diego implementation of the {\tt WSLAP+} model, so we adopt the uncertainties from the closely related Lam model.
}
\end{deluxetable*}




